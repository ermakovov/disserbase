\chapter{Обзор методов решения задачи акустического каротажа скважин}\label{ch:ch_my_1}

\section{Акустический каротаж скважин}\label{sec:ch_my_1/sec1}

Обзор методов решения задачи акустического каротажа скважин (АКС). Случай анизотропной среды. Проблемы в случае нецилиндрической геометрии.  Почему нам нужно применять постановку полного волнового обращения (ПВО) в рассматриваемом случае. Были ли примеры таких постановок до нас в АКС или других приложениях. Почему не надо бояться этого «дорогого» ПВО для АКС.  

Мы можем сделать \textbf{жирный текст} и \textit{курсив}.

\section{Случай анизотропной среды}\label{sec:ch_my_1/sec2}


\section{Нецилиндрическая геометрия скважины}\label{sec:ch_my_1/sec3}


\subsection{Подраздел}\label{subsec:ch_my_1/sec3/sub1}

